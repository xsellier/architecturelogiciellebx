% -*- mode: latex; coding: latin-1-unix -*- %

\section{R\^oles XP}
Les r\^oles XP affect\'es aux membres du groupe de projet pour cette
it\'eration sont:

\begin{itemize}
\item Architecte : Gael
\item \textsl{Coach} : Nicolas
\item Ergonome : Herv\'e
\item Testeurs : Fr\'ed\'eric et Phetsana
\item \textsl{Tracker} : Gizem
\end{itemize}

\section{T\^aches XP}
Le groupe a d\'efini avec le client la priorit\'e des t\^aches que l'on peut
classer en trois cat\'egories. Les t\^aches devront \^etre accompagn\'ees de tests.

\subsection{Fonctionnalit\'es indispensables - Priorit\'e A}
\paragraph{D\'efinition du protocole de communication} 
Le but de cette t\^ache est de d\'efinir le protocole de communication
utilis\'e entre Firefox et Laperouse
\begin{itemize}
\item D\'efinition du protocole de communication
\item R\'edaction d'une documentation
\end{itemize}

\paragraph{Cr\'eation d'un Adaptateur pour Astrolabe} 
Nous allons mettre en place:
\begin{itemize}
\item un adaptateur mono-fen\^etre
\item une documentation
\end{itemize}

\paragraph{Mise en place du protocole c\^ot\'e Firefox} 
Le but est de refaire une extension pour firefox qui permettra de
communiquer avec Laperouse via le r\'eseau.


\subsection{T\^aches importantes - Priorit\'e B}
\paragraph{Mise en place du multifen\^etre c\^ot\'e Firefox}
Gestion de plusieurs Astrolabes, un Astrolabe par onglet.

\paragraph{Mise en place du multifen\^etre c\^ot\'e Laperouse}
 Gestion de plusieurs onglets un Astrolable par onglet.


\section{D\'eroulement it\'eration (planning)}

L'enchainement des t\^aches \'etait assez lin\'eaire. Nous devions
commencer par le protocole, nous devions ensuite impl\'ementer le
monofen\^etre parrallement entre Firefox et Lap\'erouse. Nous devions
ensuite finir par l'impl\'ementation du multi client cot\'e Firefox et Lap\'erouse.


%\begin{figure}[!h]
%\begin{center}
%  \includegraphics[scale = 0.5]{gantt-cp}
%  \caption{Diagramme de squence}
%  \label{gantt-cp}
%\end{center}
%\end{figure}


\section{Bilan}
\subsection{T\^aches}
Dans cette it\'eration, nous n'avons pas pu r\'ealiser toutes les t\^aches
initialement pr\'evues. Nous avons bien s\^ur d\'efini le protocole que
nous allons utliser. Nous avons aussi mis en place le mono-fen\^etre,
mais nous n'avons malheureusement pas pu impl\'ementer le multifen\^etre.

\subsection{R\^oles}
 
\subsubsection{Architecte} Durant cette it\'eration, nous devions
int\'egrer Astrolabe dans notre projet. Il nous fallait donc remplacer
la console par Astrolabe. Pour cela nous avons cr\'e\'e un parser de
donn\'ees qui va analyser les donn\'ees et les interpr\'eter pour les
retransmettre \`a Laperouse. 
La connexion du cot\'e de Laperouse est g\'er\'ee dans le package
Socket ainsi que l'analyse des donn\'ees \'echang\'ees. 
Pour ce qui est de l'architecture g\'en\'erale de Laperouse, nous
avons gard\'e le code du groupe de l'ENSERB. Nous avons cependant
modifi\'e le Builder afin qu'il cr\'ee bien la connexion via socket. 



\subsubsection{\textsl{Coach}} 

J'ai endoss\'e le r\^ole de coach pour la derni\`ere it\'eration,
cette derni\`ere it\'eration ne n\'ecessitait pas trop d'organisation
entre les t\^aches, les grandes d\'ecisions \'etaient d\'ej\`a prises,
il ne restait plus qu' attribuer les t\^aches aux bonnes
personnes. C'est quand m\^eme au coach que revient la responsabilit\'e
des choix. Un des choix qui a \'et\'e fait a \'et\'e de tenter de
faire fonctionner correctement ce qui a \'et\'e fait plut\^ot que de
faire les autres t\^aches en ignorant les bugs existants. 

\subsubsection{Ergonome}
A l'it\'eration pr\'ec\'edente, nous avions r\'eussi \`a \'etablir une
communication \`a double sens, entre Firefox et une interface Java,
repr\'esent\'ee en l'occurrence par une simple console. Il s'agissait
pour cette it\'eration de remplacer cette console par Astrolabe. Du
point de vue de l'ergonomie, il a fallu r\ 'efl\'echir \`a la
mani\`ere la plus pratique d'utiliser Laperouse, que ce soit c\^ot\'e
Firefox ou c\^ot\'e Astrolabe. Comme il s'agissait en priorit\'e de
r\'ealiser un r\'eel \'echange d'informations entre les deux
applications, nous nous sommes concentr\'es sur une interface
utilisateur plut\^ot simplifi\'ee.\\ Ainsi, c\^ot\'e Firefox, nous
avons laiss\'e le bouton d'activation de Laperouse. D\'esormais, le
num\'ero de port choisi est sauvegard\'e dans un fichier afin d'\^etre
directement s\'electionn\'e \`a la prochaine utilisation. C\^ot\'e
Astrolabe, nous avons pens\'e \`a rajouter un menu "Laperouse"
d\'edi\'e  \`a la communication sus-cit\'ee. Par manque de temps, tous
les d\'etails d'ergonomie que nous aurions voulu impl\'ementer n'ont
pas pu tous \^etre r\'ealis\'es. 

\subsubsection{Testeurs} 
\input{test}
\input{test}

\subsubsection{\textsl{Tracker}} 
Le r\^ole de tracker a \'et\'e a tr\`es important dans cette
it\'eration, car il devait surveiller le travail et informer les
autres de l'avancement. Cela \'etait tr\`es utile ici car les binomes
travaillaient tous sur des corrections de bugs. Il \'etait donc tr\`es
important d'informer le coach de l'\'etat d'avancement afin qu'il puisse
mieux r�partir le travail restant, et la correction de nouveaux bugs
d\'ecouverts. Ce travail a \'et\'e rendu plus difficile par la
distance entre les binomes due aux vacances.


\subsection{It\'eration}

Pour cette it\'eration, le travail a \'et\'e sous-estim\'e. Nous avions
consid\'er\'e que Laperouse fonctionnait. Nous avons donc commenc\'e par
mettre en place les sockets dans le Laperouse avec l'ancien Astrolabe,
tel que cela nous a \'et\'e donn\'e par le client. Mais une fois mis en
place, nous nous sommes rendus compte que Laperouse avait encore son lot
de bugs, bugs qui nous ont beaucoup retard\'e, et que nous n'avons pas
encore r\'eussi \`a r\'esoudre. Nous avons aussi constat\'e qu'il serait tr\`es
difficile de mettre \`a jour Astrolable, sans r\'e\'ecrire une bonne partie
de Laperouse. Car le code d'Astrolabe a \'et\'e modifi\'e directement dans
certaines classes, des classes ont \'et\'e rajout\'ees dans les paquets
propres \`a Astrolabe au lieu d'\^etre dans ceux de Laperouse. Cette
it\'eration est donc inachev\'ee, celle ci constituant la derni\`ere.
